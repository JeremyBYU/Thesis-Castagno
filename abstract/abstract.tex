% MAX 550 WORDS, Exactly 570 now.
\acf{UAS} are expected to proliferate in low-altitude airspace over the coming decade. Drones with vertical takeoff and landing (VTOL) capability have been proposed to offer fast package delivery, monitor and secure assets, perform inspections, and entertain. 
Low-altitude operation of UAS in urban areas will require flight near buildings and over people. Robust urgent landing capabilities including landing site selection are required.
This thesis proposes methods to create an on-board database of maps utilized by an efficient autonomous landing site selection framework. Conventional fixed-wing emergency landing sites such as open fields and empty roadways are rare in and around cities. However, cities have many unoccupied flat rooftops that may provide nearby landing sites for small UAS. Polygons can accurately represent these flat surfaces as well as obstacles on them.

This thesis offers five specific contributions for urgent landing site selection. First, the Polylidar algorithm is developed which enables efficient non-convex polygon extraction with interior holes from 2D point sets.  A key insight of this work is a novel boundary following method that contrasts computationally expensive geometric unions of triangles. Results from real-world and synthetic benchmarks show comparable accuracy and more than four times speedup compared to other state-of-the-art methods. 

Second, we extend polygon extraction from 2D to 3D data where polygons represent flat surfaces and interior holes representing obstacles. Our Polylidar3D algorithm transforms point clouds into a triangular mesh for which planar segmentation and polygon extraction occur in parallel. Dominant plane normals are identified and used to parallelize and regularize planar segments in the mesh. Immediately after a plane has been segmented a polygon extraction task is dynamically spawned. The result is a versatile and extremely fast algorithm for non-convex polygon extraction of 3D data.

Third, we propose a framework for classifying roof shape (e.g., flat) within a city. We process satellite images, LiDAR point clouds, and map building outlines to generate both a LiDAR image and a cropped satellite image of each building. \ac{CNN}s are independently trained for each modality to extract high level features. Features from both images are fused to train a random forest classifier for final prediction. This research contributes the largest multi-city annotated dataset with over 4,500 rooftops used to train and test models. Our results show flat rooftops can identified with $> 90\%$ precision and recall.

Fourth, we integrate Polylidar3D and our roof shape prediction model to reliably extract flat rooftop surfaces from archived data sources. Our approach to surface modelling allows us take advantage of geometric methods to identify landing locations that maximize distance to obstacles. We model risk as combination of landing site and path risk metrics.  We create a multi-goal planner that guarantees a risk-optimal solution is found rapidly by avoiding exploration of high-risk options. The proposed emergency planning framework enables a UAS to select an emergency landing site and corresponding flight plan with minimum total risk.

Fifth, we verify a chosen landing site on approach by utilizing Polylidar3D aided by semantic segmentation using on-board LiDAR and camera sensors.  We construct a high-fidelity simulated city in the Unreal game engine with a statistically accurate representation of rooftop obstacles. We show that our multi-modal algorithm is robust to individual sensor failure leading to greater than 4\% improvement in IOU accuracy compared to using LiDAR only. Finally, we perform real-world flight experiments at the UMICH Flight Lab.


%  We implement parallel region growing accounting for normal orientation, Euclidean distance, and point to plane distance.  Immediately after a plane has been segmented a polygon extraction task is dynamically spawned.

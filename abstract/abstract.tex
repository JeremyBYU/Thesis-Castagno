% MAX 550 WORDS, Exactly 545 now.
Small Unmanned Aircraft Systems (sUAS) operating in low-altitude airspace require flight near buildings and over people. Robust urgent landing capabilities including landing site selection are needed. However, conventional fixed-wing emergency landing sites such as open fields and empty roadways are rare in cities. This motivates our work to uniquely consider unoccupied flat rooftops as possible nearby landing sites. We propose novel methods to identify flat rooftop buildings, isolate their flat surfaces, and find touchdown points that maximize distance to obstacles. We model flat rooftop surfaces as polygons that capture their boundaries and possible obstructions on them.

This thesis offers five specific contributions to support urgent rooftop landing. First, the Polylidar algorithm is developed which enables efficient non-convex polygon extraction with interior holes from 2D point sets.  A key insight of this work is a novel boundary following method that contrasts computationally expensive geometric unions of triangles. Results from real-world and synthetic benchmarks show comparable accuracy and more than four times speedup compared to other state-of-the-art methods. 

Second, we extend polygon extraction from 2D to 3D data where polygons represent flat surfaces and interior holes representing obstacles. Our Polylidar3D algorithm transforms point clouds into a triangular mesh where dominant plane normals are identified and used to parallelize and regularize planar segmentation and polygon extraction. The result is a versatile and extremely fast algorithm for non-convex polygon extraction of 3D data.

Third, we propose a framework for classifying roof shape (e.g., flat) within a city. We process satellite images, airborne LiDAR point clouds, and building outlines to generate both a satellite and depth image of each building. Convolutional neural networks are trained for each modality to extract high level features and sent to a random forest classifier for roof shape prediction. This research contributes the largest multi-city annotated dataset with over 4,500 rooftops used to train and test models. Our results show flat-like rooftops are identified with $> 90\%$ precision and recall. 

Fourth, we integrate Polylidar3D and our roof shape prediction model to extract flat rooftop surfaces from archived data sources. We uniquely identify optimal touchdown points for all landing sites. We model risk as an innovative combination of landing site and path risk metrics and conduct a multi-objective Pareto front analysis for sUAS urgent landing in cities. Our proposed emergency planning framework guarantees a risk-optimal landing site and flight plan is selected. 

Fifth, we verify a chosen rooftop landing site on real-time vertical approach with on-board LiDAR and camera sensors. Our method contributes an innovative fusion of semantic segmentation using neural networks with computational geometry that is robust to individual sensor and method failure.  We construct a high-fidelity simulated city in the Unreal game engine with a statistically-accurate representation of rooftop obstacles. We show our method leads to greater than 4\% improvement in accuracy for landing site identification compared to using LiDAR only.

This work has broad impact for the safety of sUAS in cities as well as Urban Air Mobility (UAM). Our methods identify thousands of additional rooftop landing sites in cities which can provide safe landing zones in the event of emergencies. However, the maps we create are limited by the availability, accuracy, and resolution of archived data. Methods for quantifying data uncertainty or performing real-time map updates from a fleet of sUAS are left for future work.
